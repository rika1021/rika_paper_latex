\usepackage[nospace]{cite}  % for smart citation
\usepackage{geometry}  % for easy margin settings
%
% margins setting
\geometry{verbose,a4paper,tmargin=3.5cm,bmargin=2cm,lmargin=4cm,rmargin=2cm}
%

% 插圖套件 graphicx
% 使用者工作流程是用 pdftex 還是 latex + dvipdfmx?
% 視情況而有不同的參數
% 這裡作自動判斷
% 參考自
% http://www.tex.ac.uk/cgi-bin/texfaq2html?label=ifpdf
%\newcommand\mydvipdfmxflow{dvipdfmx}
%\newcommand\mypdftexflow{pdftex}
%\ifx\pdfoutput\undefined
%  % not running pdftex
%  \usepackage[dvipdfm]{graphicx}
%  \newcommand\myworkflow{dvipdfmx}  % set the flag for hyperref
%\else
%  \ifx\pdfoutput\relax
%    % not running pdftex
%    \usepackage[dvipdfm]{graphicx}
%    \newcommand\myworkflow{dvipdfmx}  % set the flag
%  \else
%    % running pdftex, with...
%    \ifnum\pdfoutput>0
%      % ... PDF output
%      \usepackage[pdftex]{graphicx}
%      \newcommand\myworkflow{pdftex}  % set the flag
%    \else
%      %...DVI output
%      \usepackage[dvipdfm]{graphicx}
%      \newcommand\myworkflow{dvipdfmx}  % set the flag
%    \fi
%  \fi
%\fi

% 由於使用xelatex,所以直接搭配使用dvipdfmx
\usepackage{graphicx}

% 增強功能型頁楣 / 頁腳套件
\usepackage{fancyhdr}  % 借用此套件來擺放浮水印 
% (佔用了 central header)
% 不需要浮水印的使用者仍可利用此套件,產生所需的 header, footer
%
% 啟動 fancy header/footer 套件
\pagestyle{fancy}
\fancyhead{}  % reset left, central, right header to empty
\fancyfoot[C]{\thepage} %中間 footer 擺放頁碼
\renewcommand{\headrulewidth}{0pt} % header 的直線; 0pt 則無線

\setlength{\headheight}{15pt}


% 如果不需要任何浮水印,則請把下列介於 >>> 與 <<< 之間
% 的文字行關掉 (行首加上百分號)
%% 浮水印 >>> 
%\input{watermark/ntust_watermark.tex}
%% <<< 浮水印

% 如需額外的頁楣 (header) 或 footer,請在 my_headerfooter.tex 裡依例修改
% 它的預設內容是都關掉,可依需要打開
%\input{my_headerfooter.tex}



%%%%%%%%%%%%%%%%%%%%%%%%%%%%%%
%%%% 非必要的套件,但很實用
\usepackage{amsmath} % 各式 AMS 數學功能
\usepackage{amssymb} % 各式 AMS 數學符號
\usepackage{mathrsfs} %草寫體數學符號,在數學模式裡用 \mathscr{E} 得草寫 E
\usepackage{listings} % 程式列表套件
\usepackage{subfigure}
%
% listing setting
\lstset{breaklines=true,% 過長的程式行可斷行
extendedchars=false,% 中文處理不需要 extendedchars
texcl=true,% 中文註解需要有 TeX 處理過的 comment line, 所以設成 true
comment=[l]\%\%,% 以雙「百分號」做為程式中文註解的起頭標記,配合 MATLAB
basicstyle=\small,% 小號字體, 約 10 pt 大小
commentstyle=\upshape,% 預設是斜體字,會影響註解裏的英文,改用正體
%language=Octave % 會將一些 octave 指令以粗體顯示
}

\usepackage{url} % 在文稿中引用網址,可以用 \url{http://www.yzu.edu.tw} 方式

%%%% 以上為非必要套件
%%%%%%%%%%%%%%%%%%%%%%%%%%%%%%

%%% 以下是 hyperref 套件
%%%%%%%%%%%%%%%%%%%%%%%%%%%%%%
% hyperref 會擾亂 cite.sty 對文獻號碼縮編的排版,所以依據
% http://www.ctan.org/tex-archive/macros/latex/contrib/hyperref/
% 作如下的更動,使得 hyperref 不做文獻號碼的超連結。
\makeatletter
\def\NAT@parse{\typeout{This is a fake Natbib command to fool Hyperref.}}
\makeatother

% hyperlinkable table of contents
% 章節目錄、圖表超連結
%\ifx\myworkflow\mydvipdfmxflow
	%\usepackage[dvipdfmx, debug, colorlinks, linkcolor=black, citecolor=black, urlcolor=black, unicode]{hyperref} % remark by saiba 20110504 since dvipdfmx not supported in xelatex 
	\usepackage[debug, colorlinks, linkcolor=black, citecolor=black, urlcolor=black]{hyperref}  % remove dvipdfmx, unicode by saiba
%\else
%	\usepackage[pdftex, debug, colorlinks, linkcolor=black, citecolor=black, urlcolor=black]{hyperref}	
%\fi

% if hyperref is not used (e.g., in LyX application)
% define dummy \phantomsection for those occurences
%   in ntust_frontpages.tex, ntust_backpages.tex, my_appendix.tex
\ifx\hypersetup\undefined
	\newcommand\phantomsection{}
\fi
%%%% 以上為所有套件
%%%% 
%%%% 

% global page layout
\newcommand{\mybaselinestretch}{1.5}  %行距 1.5 倍 + 20%, (約為 double space)
\renewcommand{\baselinestretch}{\mybaselinestretch}  % 論文行距預設值
\parskip=2ex  % 段落之間的間隔為兩個 x 的高度
\parindent = 0Pt  % 段首內縮由 CJK 控制,所以這裡就設成不內縮

%%%%%%%%%%%%%%%%%%%%%%%%%%%%%
%  end of preamble
%%%%%%%%%%%%%%%%%%%%%%%%%%%%%