首先,我要向我的指導教授\underline{馮輝文} 教授致上誠摯的感謝,在這兩年多以來的碩士學業中,不只是在學術知識上給予指導,也關心學生們的身心狀況,同時期望我們不只在面對學術研究時能有嚴謹且實事求是的態度,在面對生活與做人處事上更能認真負責,抱持積極正向的心境走過遭遇的困境。更感謝教授在我輾轉找尋研究方向時,總能耐心的與我討論並指出正確的思考方向,最後能順利找到研究方向並深入探討,同時也在論文的撰寫上給予了我許多建議與方向,兩年多來的研究生活著實紮實且獲益匪淺,非常感謝教授對於我的教誨與付出的心力。

同時也要感謝口試委員\underline{林嘉慶}教授、\underline{謝宏昀}教授以及\underline{黎碧煌}教授能在繁忙的事務中抽出時間,同時感謝教授們在報告後能以不同的思維角度,給予我諸多專業且具啟發性的意見與只導,使得研究的內容能更加充實,本論文得以更臻完善。

此外,也要感謝無線通訊暨網路工程實驗室的所有成員,感謝\underline{許雅淑}學姊、\underline{雷正}學長、\underline{趙清松}學長、\underline{蔡亞宸}學姊、\underline{黃雅喻}學姊、\underline{王柏崴}學長與\underline{楊育維}學長對我的諸多照顧,總是給予許多經驗的傳授及學術知識的指導,讓我獲益良多。同時也感謝\underline{陳佳瑩}、\underline{王騰輝}與\underline{蘇毓傑}陪同一起修課,並在研究的過程中彼此給予鼓勵與安慰。也謝謝\underline{李姵瑩}、\underline{蔡雅如}、\underline{洪家琪}、\underline{劉羽軒}、\underline{曾英祖}、\underline{馬維均}、\underline{黃光緯}、\underline{鄭丁元}等學弟妹們幫忙分擔處理實驗室的各種事務,並帶來許多歡笑與輕鬆舒適的氛圍。同時也謝謝\underline{常家銘}、\underline{陳信安}、\underline{戴邵軒}、\underline{黃柏崴}、\underline{夏致群}、\underline{劉澤}、\underline{劉冠甫}等大學時的同學及學長們,在研究生生活上的各種相互扶持與幫助。

最後,要感謝我的家人對我的照顧,不時給予我精神的鼓勵與物質上的支持,讓我能在求學過程中能專心進行研究,也謝謝女友\underline{琳達}一直以來的陪伴,互相分享生活上發生的大小事,即使在外地工作仍能感受心中的溫暖。感謝所有幫助過我的人,若非有你們,我無法獨自完成今日的成果。願家人和所有愛我與我愛的人,能一同分享碩士研究與論文完成的喜悅。


\begin{flushright}
李承洋\quad 謹誌於 2016/12/19 臺灣科大 無線通訊暨網路工程實驗室RB306-2
\end{flushright}
%\raggedleft 