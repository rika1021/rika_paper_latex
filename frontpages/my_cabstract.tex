
由於目前長期演進上行傳輸使用的經典排程演算法並未顧及使用者裝置的延遲狀況且其公平性不佳,導致部分使用者裝置得不到資源以傳輸資料且封包嚴重遺失,再者,資源分配方式的效率亦不佳,因此,本碩士論文著眼於設計一排程演算法來消弭封包大量遺失的狀況並提升使用者裝置的公平性。為能改善封包遺失率,我們的排程演算法不以使用者裝置通道品質為基礎,而是加入延遲預算做考量,選擇使用者裝置中平均佇列頭端封包剩餘延遲預算最少的使用者裝置為優先對象,使其能優先獲得較佳的資源以進行資料傳輸,降低使用者裝置的封包遺失率。除此之外,透過參考使用者裝置平均配置過的資源數量,讓取得過較少資源的使用者裝置也能有機會取得資源,進而提升整體的公平性,另外,我們的排程演算法在連續分配上設置分配數量限制,能更有效地利用資源,進一步提升系統的吞吐量。最後,在固定服務流數量與隨機服務流數量安排下,模擬數據印證本碩士論文提出之排程演算法均能達到預期的效果,與其他相關的排程演算法相比,除了能有效降低封包的遺失率,更可提升整體使用者裝置彼此間的公平性及系統的吞吐量。
\\
\\
關鍵字:長期演進,上行排程演算法,封包延遲,延遲預算,封包遺失率。
%我們在排程演算法的設計上加入延遲預算做為主要考量,優先分配資源給封包延遲時間逼近延遲預算的使用者,讓排程器能優先選取較迫切需要得到資源的使用者,使其能優先獲得較佳的資源以進行資料傳輸,而其他經典的排程演算法在沒有考量延遲預算的情況下,通常以較佳通道品質為主要考量的資源分配方式,容易使通道品質較差的使用者得不到資源,導致封包因超過其最大延遲時間而被丟棄,造成封包遺失率上升。除此之外,為了在連續分配上能更有效的利用資源,進一步提升系統的吞吐量,我們的排程演算法能(透過什麼原因/方式/作法)讓通道品質較差的使用者也能有機會取得資源,進而提升整體的公平性。