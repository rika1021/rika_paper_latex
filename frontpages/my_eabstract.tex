
Nowadays, classic scheduling algorithms used in LTE uplink fail to take care of delay conditions and fairness among user equipments(UEs), resulting in severe packet losses and poor fairness for some UEs. Meanwhile, the corresponding resource allocation is not efficient as well. Therefore, we aim at designing a scheduling algorithm to avoid packet losses and enhance fairness. Targeting towards the aforementioned goals, the delay budget rather than the channel condition of UEs serves as the key factor when designing our algorithm. Our algorithm employs the principle that the UE with the least delay budget left in queue will be scheduled preferentially first. By doing so, it is allowed to get good-quality resources to transmit data so that the problem of packet losses can be alleviated. Furthermore, the amount of allocated resources is taken into account. UEs with fewer allocated resources still have chances to be served to improve fairness. In addition, our proposed algorithm poses a constraint in sequential resource allocation so that resources can be utilized efficiently and the overall throughput can be improved. Finally, our simulation results support that our proposed algorithm performs well as expected either in a scenario with a fixed number of service flows or in a scenario with a random number of service flows. Compared with the other related algorithms, our proposed algorithm has fewer packet losses as well as better system throughput.
\\
\\
Keywords:LTE, Uplink Scheduling, Packet Delay, Delay Budget, Packet Loss Ratio.
%having packets with shorter time left in queue to be scheduled and sets constraint in sequential allocation method.
%Under circumstances that reliability and penetration of intellegent portable devices rises with technology in modern society, efficient scheduling with limited resources plays an important role.
% To maximize system throughput, algorithms widely used in uplink regard users with good channel conditions as primary resource allocation objects. Those with poor channel conditions are lack of resources and drop packets frequently that results in high packet loss ratio, poor throughput and fairness. The method of sequential resource block allocation effects the throughput as well.