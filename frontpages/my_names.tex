%
% this file is encoded in utf-8
% v2.0 (Apr. 5, 2009)
% 填入你的論文題目、姓名等資料
% 如果題目內有必須以數學模式表示的符號,請用 \mbox{} 包住數學模式,如下範例
% 如果中文名字是單名,與姓氏之間建議以全形空白填入,如下範例
% 英文名字中的稱謂,如 Prof. 以及 Dr.,其句點之後請以不斷行空白~代替一般空白,如下範例
% 如果你的指導教授沒有如預設的三位這麼多,則請把相對應的多餘教授的中文、英文名
%    的定義以空的大括號表示
%    如,\renewcommand\advisorCnameB{}
%          \renewcommand\advisorEnameB{}
%          \renewcommand\advisorCnameC{}
%          \renewcommand\advisorEnameC{}

% 論文題目 (中文)
\renewcommand\cTitle{%
採用深度強化學習結合時間序列神經網路預測模型的自適性交通信號系統設計
}

% 論文題目 (英文)
\renewcommand\eTitle{%
Design of Adaptive Traffic Light Signaling System Using Deep Reinforcement Learning combined with Time Series Neural Network Forecasting Model}

% 我的姓名 (中文)
\renewcommand\myCname{吳宜真}

% 我的姓名 (英文)
\renewcommand\myEname{Yi-Chen Wu}

% 我的學號 (英文)
\renewcommand\myStudentID{M10815045}

% 指導教授A的姓名 (中文)
\renewcommand\advisorCnameA{馮輝文 \quad 博士}

% 指導教授A的姓名 (英文)
\renewcommand\advisorEnameA{Dr. Huei-Wen Ferng}

% 指導教授B的姓名 (中文)
\renewcommand\advisorCnameB{}

% 指導教授B的姓名 (英文)
\renewcommand\advisorEnameB{}

% 指導教授C的姓名 (中文)
\renewcommand\advisorCnameC{}

% 指導教授C的姓名 (英文)
\renewcommand\advisorEnameC{}

% 校名 (中文)
\renewcommand\univCname{國立臺灣科技大學}

% 校名 (英文)
\renewcommand\univEname{National Taiwan University of Science and Technology}

% 系所名 (中文)
\renewcommand\deptCname{資訊工程系}

% 系所全名 (英文)
\renewcommand\fulldeptEname{Department of Computer Science and Information Engineering}

% 系所短名 (英文, 用於書名頁學位名領域)
%\renewcommand\deptEname{Electro-Optical Engineering}

% 學院英文名 (如無,則以空的大括號表示)
\renewcommand\collEname{College of Electrical Engineering and Computer Science}

% 學位名 (中文)
\renewcommand\degreeCname{碩士}

% 學位名 (英文)
\renewcommand\degreeEname{Master of Science}

% 口試年份 (中文、民國)
\renewcommand\cYear{一百一十一}

% 口試月份 (中文)
\renewcommand\cMonth{十} 

% 口試日期 (中文)
\renewcommand\cDay{十九}

% 口試年份 (阿拉伯數字、西元)
\renewcommand\eYear{2022} 

% 口試月份 (英文)
\renewcommand\eMonth{October}

% 學校所在地 (英文)
%\renewcommand\ePlace{Taipei City, Taiwan}

%畢業級別;用於書背列印;若無此需要可忽略
\newcommand\GraduationClass{103}

%%%%%%%%%%%%%%%%%%%%%%