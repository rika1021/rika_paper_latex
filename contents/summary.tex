%
% this file is encoded in utf-8
% v1.7
\chapter{結論}
於本碩士論文中,我們針對降低封包遺失率以及確保公平性,提出一適用於長期演進上行網路的封包排程演算法UFS。不同於以往為達系統最大吞吐量的設計因只顧及使用者裝置的通道品質,造成封包遺失嚴重與公平性低落的問題,除了考量通道品質外,我們更考量了使用者裝置中的佇列頭端封包延遲,其優先於通道品質做為決定使用者裝置分配資源的順序,將佇列頭端封包延遲逼近其延遲預算時間的使用者裝置視為目前急需資源的對象,給予其較高的優先權以取得通道品質較佳的通道進行傳輸;除此之外亦加入平均配置資源區塊數量做為部分優先度權值,來維持使用者裝置之間的公平性;最後在資源區塊的連續分配上也加入調變編碼技術限制,以維持同等級的調變編碼技術來限制資源區塊分配的數量,避免因分配大量資源區塊數量卻不能達到預期能傳輸的資料量,去除占用過多資源的疑慮。而模擬結果支持了我們在排程演算法的設計方針,反應出預期的效果,不論在固定或隨機服務流數量的環境下,UFS在封包遺失率或是公平性上都較其他排程演算法優異。不過因為考量了佇列頭端封包延遲,會讓使用者裝置即使有較好的通道品質,也需要將資源優先給急需的使用者裝置而被迫等待,使得整體的延遲高於其他排程演算法,不過在各種不同種類的服務流都未超過其最大延遲時間上限;同時於系統吞吐量的表現上,在較貼近現實的隨機服務流數量的環境下也優於其他排程演算法。綜合以上在各面相的表現上,僅管在端點對端點延遲上表現稍嫌不佳,但仍在可以容忍的範圍內,各服務流也都並未超過最大延遲時間,且在其他方面則均有較佳的表現,故總體來看UFS相較於其他排程演算法可以達到降低封包遺失率及提升公平性,並達到較多的吞吐量之效果。

在未來我們可以在計算剩餘延遲預算上同時考慮服務流種類的考量,利用加權過的剩餘延遲預算能更細緻反應每個使用者裝置的剩餘延遲預算狀況。另外,也能考慮改善分配量門檻值的決定方式,週期性地利用需要排程的使用者裝置數量,使用動態調整的方式決定分配量門檻值,使得調變編碼技術限制的啟動能更有彈性,或者在連續分配時,簡化方向性的判斷,更優化排程演算法。